\documentclass[12pt]{article}
\usepackage[utf8]{inputenc}
\usepackage[english]{babel}
\usepackage{csquotes}
\usepackage[T1]{fontenc}
\usepackage{lmodern}
\usepackage[
    backend=biber,
    style=numeric,
    citestyle=numeric,
    sorting=none
]{biblatex}
\addbibresource{derivations.bib}
% dcases
\usepackage{mathtools}
% for text right arrow
\usepackage{textcomp}
\usepackage{amssymb}
\usepackage{amsmath}
\usepackage[breaklinks=true, linkcolor=blue, citecolor=blue, colorlinks=true]{hyperref}
% load after hyperref
\usepackage[english,capitalise]{cleveref}

\newcommand{\pluseq}{\mathrel{{+}{=}}}
\newcommand{\minuseq}{\mathrel{{-}{=}}}
\newcommand{\ns}{\ensuremath{{N_{sp}}}}
\newcommand{\nr}{\ensuremath{{N_{reac}}}}
\newcommand{\conp}{CONP}
\newcommand{\conv}{CONV}
\newcommand{\dconp}{\ensuremath{,\qquad\text{for \conp}}}
\newcommand{\dconv}{\ensuremath{,\qquad\text{for \conv}}}
\newcommand{\Ru}{\ensuremath{\mathcal{R}}}

%fix to dcases from here:http://tex.stackexchange.com/questions/252410/centering-in-dcases-environment/252414
\MHInternalSyntaxOn
\renewcommand{\dcases}
 {
  \MT_start_cases:nnnn
    {\quad}
    {$\m@th\displaystyle##$\hfil}
    {$\m@th\displaystyle##$\hfil}
    {\lbrace}
 }
\MHInternalSyntaxOff

\begin{document}
\section{Introduction}
This document contains the complete derivations for \texttt{pyJac} v2.0, which generates code to evaluate the chemical source term and analytical chemical kinetic Jacobians of constant-pressure\slash constant-volume, fixed-mass, adiabatic reactors.
Note that equations specific to the constant-pressure or constant-volume derivation will be marked with \conp~or \conv~respectively.
The derivations in this script based upon the output of the script \href{https://github.com/arghdos/SPyJac-paper/blob/master/derivations/scripts/derivations.py}{derivations.py} in the GitHub repository for this paper.
This script depends on the symbolic mathematics library SymPy~\cite{sympy}, and was tested with versions 1.0 and 1.1.1

\section{Governing equations}
\subsection{State variables}
The state vector for this derivation consists of the temperature, a thermodynamic parameter (pressure or volume) and the amount of moles of all species except the last in the model:

\begin{subequations}
\begin{align}
\Phi &= \left\{T, V, n_1, n_2 \ldots n_{\ns - 1}\right\}\dconp, \\
\Phi &= \left\{T, P, n_1, n_2 \ldots n_{\ns - 1}\right\}\dconv
\end{align}
\end{subequations}
where $T$ is the temperature, $P$ and $V$ the pressure and volume respectively, and $n_j$ the number of moles of the $j$th species in the model (containing $\ns$ total species).

From the ideal gas law, the number of moles of the final species is determined as:
\begin{align}
n &= \frac{V P}{T \Ru} = \sum_{i=1}^{\ns}{n_i}, \label{e:source_moles}\\
n_{\ns} &= \frac{V P}{T \Ru} - \sum_{i=1}^{\ns - 1}{n_i}
\end{align}

\subsection{Thermo-chemical source terms}
The evolution of the thermo-chemical state variables of this system is described by a set of ordinary-differential equations:
\begin{subequations}
\label{e:source_terms}
\begin{align}
f &= \frac{\text{d} \Phi }{\text{d} t } = \left\{\frac{\text{d} T }{\text{d} t },\frac{\text{d} V }{\text{d} t },\frac{\text{d} n_1}{\text{d} t },\frac{\text{d} n_2 }{\text{d} t }\ldots \frac{\text{d} n_{\ns - 1} }{\text{d} t }\right\}\dconp, \\
f &= \frac{\text{d} \Phi }{\text{d} t } = \left\{\frac{\text{d} T }{\text{d} t },\frac{\text{d} P }{\text{d} t },\frac{\text{d} n_1}{\text{d} t },\frac{\text{d} n_2 }{\text{d} t }\ldots \frac{\text{d} n_{\ns - 1} }{\text{d} t }\right\}\dconv 
\end{align}
\end{subequations}
For both, the molar source terms are~\cite{TurnsStephenR2012Aitc}:
\begin{equation}
\frac{\text{d} n }{\text{d} t }_{k} = V \dot{\omega}_{k} \label{e:spec}
\end{equation}
where $\dot{\omega}_k$ is the $kth$ species overall production rate, and the temperature production rates are~\cite{TurnsStephenR2012Aitc}:
\begin{subequations}
\label{e:temperature_incomplete}
\begin{align}
\frac{\text{d} T }{\text{d} t } &= - \frac{\sum_{k=1}^{\ns} H_{k} \dot{\omega}_{k}}{\sum_{k=1}^{\ns} [C]_{k} {C_{p, k}}}\dconp, \\
\frac{\text{d} T }{\text{d} t } &= - \frac{\sum_{k=1}^{\ns} U_{k} \dot{\omega}_{k}}{\sum_{k=1}^{\ns} [C]_{k} {C_{v, k}}}\dconv
\end{align}
\end{subequations}
where $H_k$, $U_k$, $C_{p,k}$ and $C_{v, k}$ are the enthalpy, internal energy, constant-pressure specific heat, constant-volume specific heat of species $k$ in molar units, while $[C]_{k}$ is the concentration, given by:
\begin{equation}
 [C]_{k} = \frac{n_{k}}{V}
\end{equation}


From the ideal gas law, source terms for the volume and pressure variables may derived:
\begin{subequations}
\label{e:param_incomplete}
\begin{align}
\frac{\text{d} V }{\text{d} t } &= \frac{\Ru}{P} \left(T \frac{\text{d} n }{\text{d} t } + \frac{\text{d} T }{\text{d} t } n\right)\dconp, \\
\frac{\text{d} P }{\text{d} t } &= \frac{\Ru}{V} \left(T \frac{\text{d} n }{\text{d} t } + \frac{\text{d} T }{\text{d} t } n\right)\dconv
\end{align}
\end{subequations}

To determine $\frac{\text{d} n }{\text{d} t }$, conservation of mass is invoked:
\begin{equation}
 \frac{\text{d} m }{\text{d} t } = 0 = \sum_{k=1}^{\ns}  W_{k} \frac{\text{d} n }{\text{d} t }_{k},
\end{equation}
from which, the source term of the last species may be written in terms of the species included in the state vector:
\begin{equation}
 \frac{\text{d} n }{\text{d} t }_{\ns} = - \frac{1}{W_{\ns}} \sum_{k=1}^{\ns - 1} W_{k} \frac{\text{d} n }{\text{d} t }_{k}
 \label{e:spec_ns}
\end{equation}
where $W_{k}$ is the molecular weight of the $k$th species.
Using~\cref{e:spec_ns}, the total molar rate of change can be determined:
\begin{align}
\frac{\text{d} n }{\text{d} t } &= \sum_{k=1}^{\ns} \frac{\text{d} n }{\text{d} t }_{k}, \nonumber \\
\frac{\text{d} n }{\text{d} t } &= \sum_{k=1}^{\ns - 1} \left(1 - \frac{W_{k}}{W_{\ns}}\right) \frac{\text{d} n }{\text{d} t }_{k}
\label{e:total_molar}
\end{align}

Combining~\cref{e:spec,e:param_incomplete,e:total_molar} gives the final form of the pressure and volume terms:
\begin{subequations}
\label{e:param_complete}
\begin{align}
\frac{\text{d} V }{\text{d} t } &= V \left(\frac{T \Ru}{P} \sum_{k=1}^{-1 + \ns} \left(1 - \frac{W_{k}}{W_{\ns}}\right) \dot{\omega}_{k} + \frac{1}{T} \frac{\text{d} T }{\text{d} t }\right)&\dconp, \\
\frac{\text{d} P }{\text{d} t } &= \frac{P}{T} \frac{\text{d} T }{\text{d} t } + T \Ru \sum_{k=1}^{-1 + \ns} \left(1 - \frac{W_{k}}{W_{\ns}}\right) \dot{\omega}_{k}&\dconv
\end{align}
\end{subequations}

Additionally, the concentration and moles of the last species in the model (which is not a state variable) may be expanded in terms of the state variables of the system:
\begin{equation}
 \label{e:last_spec_conc}
   [C]_{\ns} = [C] - \sum_{k=1}^{-1 + \ns} [C]_{k}
\end{equation}
where $[C]$ is the total concentration:
\begin{equation}
 [C] = \frac{P}{T \Ru}
\end{equation}

Using~\cref{e:last_spec_conc,e:spec,e:spec_ns}, the temperature source terms may be expanded to contain only state variables (i.e., by removal of the last species' source term and overall production rate):
\begin{subequations}
\label{e:temperature_complete}
\begin{align}
\frac{\text{d} T }{\text{d} t } = - \frac{\sum_{k=1}^{-1 + \ns} \left(H_{k} - \frac{W_{k} H_{\ns}}{W_{\ns}}\right) \dot{\omega}_{k}}{[C] {C_{p,\ns}} + \sum_{k=1}^{-1 + \ns} \left(- {C_{p,\ns}} + {C_{p, k}}\right) [C]_{k}}\dconp,\\
\frac{\text{d} T }{\text{d} t } = - \frac{\sum_{k=1}^{-1 + \ns} \left(U_{k} - \frac{W_{k} U_{\ns}}{W_{\ns}}\right) \dot{\omega}_{k}}{[C] {C_{v,\ns}} + \sum_{k=1}^{-1 + \ns} \left(- {C_{v,\ns}} + {C_{v, k}}\right) [C]_{k}}\dconv
\end{align}
\end{subequations}
This form is used for differentiation, but~\cref{e:temperature_incomplete} will often be used as well due to its compactness.

\subsection{Thermal properties}
The standard-state thermodynamic properties (in molar units) for a gaseous species $k$ is defined using the standard seven-coefficient polynomial of Gordon and McBride~\cite{gordon1994computer}:
\begin{align}
\frac{C_{p,k}^{\circ}}{\mathcal{R}} &= a_{0,k} + T \left( a_{1,k} + T \left( a_{2,k} + T \left( a_{3,k} + a_{4,k} T \right) \right) \right) \label{e:cpk} \\
\frac{H_k^{\circ}}{\mathcal{R}} &= T \left( a_{0,k} + T \left( \frac{a_{1,k}}{2} + T \left( \frac{a_{2,k}}{3} + T \left( \frac{a_{3,k}}{4} + \frac{a_{4,k}}{5} T \right) \right) \right) \right) + a_{5,k} \label{e:hk} \\
\frac{S_k^{\circ}}{\mathcal{R}} &= a_{0,k} \ln T + T \left( a_{1,k} + T \left( \frac{a_{2,k}}{2} + T \left( \frac{a_{3,k}}{3} + \frac{a_{4,k}}{4} T \right) \right) \right) + a_{6,k} \label{e:sk}
\end{align}
where $C_{p,k}$ and $H_k$ are as described previously, $S_k$ is the entropy in molar units, and the ${}^{\circ}$ indicates a standard-state property at one atmosphere (equivalent to the property at any pressure for calorically perfect gases).

\subsection{Reaction rate expressions}
The net species rate of production is defined as:
\begin{equation}
 \label{e:spec_rop}
 \dot{\omega}_{k} = \sum_{i=1}^{\nr} \nu_{k,i} q_{i}
\end{equation}
where $\nr$ is the number of reactions in the chemical kinetic model, $\nu_{k, i}$ the overall stoichiometric coefficient of species $k$ in reaction $i$ and $q_i$ the net rate-of-progress for reaction $i$:
\begin{align}
\nu_{k,i} &= \nu^{\prime\prime}_{k,i} - \nu^{\prime}_{k,i} \\
q_{i} &= R_{i} c_{i}
\end{align}
with $\nu^{\prime}_{k,i}$ and $\nu^{\prime\prime}_{k,i}$ the product and reactant stoichiometric cofficients (respectively) of species $k$ in reaction $i$.
The base rate-of-progress for the $i$th reversible reaction $R_{i}$ is given by:
\begin{align}
R_{i} &= {R_f}_{i} - {R_r}_{i} \label{e:ropnet}\\
{R_f}_{i} &= {k_f}_{i} \prod_{k=1}^{\ns} [C]_{k}^{\nu^{\prime}_{k,i}} \label{e:ropf}\\
{R_r}_{i} &= {k_r}_{i} \prod_{k=1}^{\ns} [C]_{k}^{\nu^{\prime\prime}_{k,i}} \label{e:ropr}
\end{align}
where ${k_f}_{i}$ and ${k_r}_{i}$ are the forward and reverse reaction coefficients (respectively) for the $i$th reaction, and the third-body\slash pressure modification $c_{i}$ is given by:
\begin{equation}
\label{e:rxn_pressure}
c_i = \begin{dcases}
  1 &\text{for elementary reactions,} \\
  [X]_i &\text{for third-body enhanced reactions,} \\
  \frac{P_{r,i}}{1 + P_{r,i}} F_i &\text{for unimolecular/recombination falloff reactions, and} \\
  \frac{1}{1 + P_{r,i}} F_i &\text{for chemically-activated bimolecular reactions,}
  \end{dcases}
\end{equation}
where for the $i$th reaction $[X]_i$ is the third-body concentration, $P_{r,i}$ is the reduced pressure, and $F_i$ is the falloff blending factor.
These terms are defined in the following sections.

The forward reaction rate coefficient $k_{f, i}$ is given by the three-parameter Arrhenius expression:
\begin{equation}
  \label{e:arrhenius}
  {k_f}_{i} = A_{i} T^{\beta_{i}} \operatorname{exp}\left({- \frac{{E_{a}}_{i}}{T \Ru}}\right) \;,
\end{equation}
where $A_i$ is the pre-exponential factor, $\beta_i$ is the temperature exponent, and $T_{a, i}$ is the activation temperature given by $T_{a, i} = E_{a, i} / \mathcal{R}$.

As given by Lu and Law~\cite{Lu:2009gh}, depending on the value of the Arrhenius parameters, $k_{f,i}$ can be calculated in different ways to minimize the computational cost:
\begin{equation}
  k_{f,i} =
  \begin{dcases}
  A_i & \text{if } \beta = 0 \text{ and } T_{a,i} = 0 \;, \\
  \exp \left( \log A_i + \beta_i \log T \right)   & \text{if } \beta_i \neq 0 \text{ and } T_{a, i} = 0 \;, \\
  \exp \left( \log A_i + \beta_i \log T - T_{a, i} / T \right) & \text{if } \beta_i \neq 0 \text{ and } T_{a, i} \neq 0 \;, \\
  \exp \left( \log A_i - T_{a, i} / T \right)  & \text{if } \beta_i = 0 \text{ and } T_{a, i} \neq 0 \;, \text{ and} \\
  A_i \prod^{\beta_i} T & \text{if } T_{a, i} = 0 \text{ and } \beta_i \in \mathbb{Z} \;,
  \end{dcases}
\end{equation}
where $\mathbb{Z}$ is the set of integers; the extent of this specialization can be controlled when generating code via \texttt{pyjac}.

\subsection{Reverse rate coefficient}
By definition, the reverse rate coefficient of irreversible reactions ${k_r}_{i}$ is zero, while reversible reactions may have non-zero ${k_r}_{i}$.
Note that in \texttt{pyJac}, reversible reactions with explicit reverse Arrhenius parameters are split into two irreversible reactions; this simplifies calculation inside the generated code, and eases comparison to Cantera~\cite{Goodwin:2015aa} which applies the same transformation.
For reversible reactions without an explicit parameterization, the reverse rate coefficient is calculated from ratio of the forward rate coefficient and the equilibrium constant:
\begin{align}
 {k_r}_{i} &= \frac{{k_f}_{i}}{{K_c}_{i}}\; , \label{e:kr}\\
 {K_c}_{i} &= \left(\left(\frac{P_{atm}}{T \Ru}\right)^{\sum_{k=1}^{\ns} \nu_{k,i}}\right) {K_p}_{i}\; ,\text{ and} \label{e:kc}\\
 {K_p}_{i} &= \text{exp}\left(\frac{\Delta S^{\circ}_k}{\Ru} - \frac{\Delta H^{\circ}_k}{\Ru T}\right) = \text{exp}\left(\sum_{k=1}^{\ns}\nu_{ki}\left(\frac{S^{\circ}_k}{\Ru} - \frac{H^{\circ}_k}{\Ru T}\right)\right) \label{e:kp}
\end{align}
where $P_{atm}$ is the pressure of one standard atmosphere in appropriate units.

By combining~\cref{e:kc,e:kp}, we obtain:
\begin{equation}
 \label{e:kc_in_kp}
 {K_c}_{i} = \left(\left(\frac{P_{atm}}{\Ru}\right)^{\sum_{k=1}^{\ns} \nu_{k,i}}\right) \operatorname{exp}\left({\sum_{k=1}^{\ns} \nu_{k,i} B_{k}}\right)
\end{equation}
where, $B_k$ is:
\begin{align}
 \label{e:Bk}
 B_{k} &= \frac{S^{\circ}_k}{\Ru} - \frac{H^{\circ}_k}{\Ru T} - \ln{T} \nonumber\; , \\
 B_{k} &= T \left(T \left(T \left(\frac{T a_{k,4}}{20} + \frac{a_{k,3}}{12}\right) + \frac{a_{k,2}}{6}\right) + \frac{a_{k,1}}{2}\right) \nonumber \\
       & \quad + \left(a_{k,0} - 1\right) \log{\left (T \right )} - a_{k,0} + a_{k,6} - \frac{a_{k,5}}{T}
\end{align}
from~\cref{e:hk,e:sk}.

\subsection{Third-body effects}
\label{s:thdbody}

For a reaction enhanced (or diminished) by the presence of a third body, the reaction rate is modified by the third-body concentration $[X]_i$ given by
\begin{equation}
[X]_{i} = \sum_{k=1}^{\ns} \alpha_{k,i} [C]_{k} \;,
\end{equation}
where $\alpha_{k,i}$ is the third-body efficiency of species $k$ in the $i$th reaction.
For a default third-body efficiency of $\alpha_{k,i} = 1$, this may be rearranged to the compact-storage form:
\begin{equation}
 [X]_{i} = [C] + \sum_{k=1}^{\ns} \left(\alpha_{k,i} - 1\right) [C]_{k} \;,
\end{equation}
where only species with non-default third-body efficiencies must be stored in the model.
Expanding the concentration of the last species gives:
\begin{equation}
\label{e:thd_mix}
 [X]_{i}=[C] \alpha_{\ns,i} + \sum_{k=1}^{-1 + \ns} \left(- \alpha_{\ns,i} + \alpha_{k,i}\right) [C]_{k}\;.
\end{equation}
If all species in the mixture contribute equally as third bodies, then $\alpha_{k,i} = 1$ for all species:
\begin{equation}
\label{e:thd_unity}
 [X]_{i} = [C] = \frac{P}{\Ru T} \;.
\end{equation}
In addition, a single species may act as the third body in which case
\begin{equation}
 [X]_{i} = [C_m]
\end{equation}
where the $m$th species is the third body.
For evaluation of Jacobian entries, the expanded form:
\begin{equation}
\label{e:thd_spec}
 [X]_{i}=\left([C] - \sum_{k=1}^{-1 + \ns} [C]_{k}\right) \delta_{\ns m} + \left(- \delta_{\ns m} + 1\right) [C]_{m}
\end{equation}
will be used, where the Kronecker delta $\delta_{\ns m}$ is unity if and only if the third body species $m$ is the last species in the model.


\subsection{Falloff reactions}
Unlike elementary and third-body reactions, falloff reactions exhibit a pressure dependence described as a blending of rates at low- and high-pressure limits; thus, the rate coefficients depend on a mixture of low-pressure- ($k_{0, i}$) and high-pressure-limit ($k_{\infty,i}$) coefficients, each with corresponding Arrhenius parameters and expressed using Eq.~\cref{e:arrhenius}.
The ratio of the coefficients $k_{0, i}$ and $k_{\infty, i}$, combined with the third-body concentration (based on either the mixture as a whole including any efficiencies $\alpha_{k,i}$, or a specific species), define a reduced pressure $P_{r,i}$ given by
\begin{equation}
 P_{r, i}=\frac{[X]_{i} k_{0, i}}{k_{\infty, i}}
\end{equation}
where $[X]_{i}$ is the appropriate third-body concentration as described in~\cref{s:thdbody}

The falloff blending factor $F_i$ used in Eq.~\cref{e:rxn_pressure} is determined based on one of three representations: the Lindemann~\cite{Lindemann:1922cz}, Troe~\cite{Gilbert:1983bb}, and SRI~\cite{Stewart:1989gj} falloff approaches
\begin{equation}
F_i = \begin{dcases}
1 &\text{for Lindemann,} \\
F_{\text{cent}}^{ \left( 1 + ( A_{\text{Troe}} / B_{\text{Troe}} )^2 \right)^{-1} } &\text{for Troe, or} \\
d T^e \left( a \cdot \exp \left( -\frac{b}{T} \right) + \exp \left( -\frac{T}{c} \right) \right)^X &\text{for SRI.}
\end{dcases}
\end{equation}
The Troe representation is described by the variables:
\begin{align}
 F_{cent} &= a \operatorname{exp}\left({- \frac{T}{T^{*}}}\right) + \left(- a + 1\right) \operatorname{exp}\left({- \frac{T}{T^{***}}}\right) + \operatorname{exp}\left({- \frac{T^{**}}{T}}\right) \;, \\ 
 A_{Troe} &= - 0.67 \log_{10}{\left (F_{cent} \right )} + \log_{10}{\left (P_{r, i} \right )} - 0.4 \;\text{, and}\\
 B_{Troe} &= - 1.1762 \log_{10}{\left (F_{cent} \right )} - 0.14 \log_{10}{\left (P_{r, i} \right )} + 0.806 
\end{align}
where $a$, $T^{***}$, $T^*$, and $T^{**}$ are specified parameters.
The final parameter $T^{**}$ is optional, and, if it is not used, the final term of $F_{\text{cent}}$ is omitted.

The exponent used in the SRI representation is given by:
\begin{equation}
 X = \frac{1}{\log_{10}^{2}{\left (P_{r, i} \right )} + 1}
\end{equation}
where $a$, $b$, and $c$ are required parameters.
The parameters $d$ and $e$ are optional; if not specified, $d = 1$ and $e = 0$.

\subsection{Pressure-dependent reactions}

In addition to the falloff approach given previously, two additional formulations can be used to describe the pressure dependence of reactions that do not follow the modification factor $c_i$ approach.
The first involves logarithmic interpolation between Arrhenius rates at two pressures~\cite{chemkin:2012,Goodwin:2015aa}, each evaluated using Eq.~\cref{e:arrhenius}:
\begin{align}
k_1 (T) &= A_1 T^{\beta_1} \exp \left( -\frac{T_{a, 1}}{T} \right) \text{ at } P_1 \text{ and} \label{e:plog_k1} \\
k_2 (T) &= A_2 T^{\beta_2} \exp \left( -\frac{T_{a, 2}}{T} \right) \text{ at } P_2 \;, \label{e:plog_k2}
\end{align}
where the Arrhenius coefficients are given for each pressure $p_1$ and $p_2$.
Then, the reaction rate coefficient at a particular pressure $p$ between $p_1$ and $p_2$ can be determined through logarithmic interpolation:
\begin{equation}
\log{\left ({k_f}_{i} \right )} = \frac{\left(- \log{\left (k_{1} \right )} + \log{\left (k_{2} \right )}\right) \left(- \log{\left (P_{1} \right )} + \log{\left (P \right )}\right)}{- \log{\left (P_{1} \right )} + \log{\left (P_{2} \right )}} + \log{\left (k_{1} \right )}
\end{equation}

In addition, the pressure dependence of a reaction can be expressed through a bivariate Chebyshev polynomial fit~\cite{Venkatesh:1997hv,Venkatesh:1997ik,Venkatesh:2000gj,chemkin:2012,Goodwin:2015aa}:
\begin{equation}
\log_{10} k_f (T, p) = \sum_{i = 1}^{N_T} \sum_{j = 1}^{N_p} \eta_{ij} \phi_i (\tilde{T}) \phi_j \left(\tilde{p}\right) \label{e:cheb} \;,
\end{equation}
where $\eta_{ij}$ is the coefficient corresponding to the grid points $i$ and $j$, $N_T$ and $N_p$ are the numbers of grid points for temperature and pressure, respectively, and $\phi_n$ is the Chebyshev polynomial of the first kind of degree $n - 1$ typically expressed as
\begin{equation}
\phi_n (x) = \mathcal{T}_{n-1} (x) = \cos \left( (n - 1) \arccos (x) \right) \quad \text{for } |x| \leq 1 \;.
\end{equation}
The reduced temperature $\tilde{T}$ and pressure $\tilde{p}$ are given by
\begin{align}
\tilde{T} &\equiv \frac{2 T^{-1} - T^{-1}_{\min} - T^{-1}_{\max}}{T^{-1}_{\max} - T^{-1}_{\min}} \quad\text{and} \\
\tilde{p} &\equiv \frac{2\log_{10} p - \log_{10} p_{\min} - \log_{10} p_{\max}}{\log_{10} p_{\max} - \log_{10} p_{\min}} \;,
\end{align}
where $T_{\min} \leq T \leq T_{\max}$ and $p_{\min} \leq p \leq p_{\max}$ describe the ranges of validity for temperature and pressure.

\section{Jacobian derivation}

Let $\mathcal{J}$ denote the Jacobian matrix corresponding to the set of ODEs defined in~\cref{e:source_terms}.
$\mathcal{J}$ is filled with the partial derivatives $\partial f / \partial \Phi$, such that:
\begin{equation}
 \mathcal{J}_{i,j} = \frac{\partial f_i}{\partial \Phi_j},\qquad j=1 \ldots \ns - 1
\end{equation}
We will begin by deriving subcomponents of the Jacobian matrix in the subsequent sections before presenting the final form of the Jacobian matrix in~\cref{s:jac_final}.

\subsection{Species rate of production derivatives}
First, we evaluate the derivatives of the net species rate of production equation~\cref{e:spec_rop}, with respect to temperature:
\begin{equation}
 \label{e:dwdot_dT}
 \frac{\partial \dot{\omega} }{\partial T }_{k} = \sum_{i=1}^{\nr} \left(\nu_{k,i} R_{i} \frac{\partial c }{\partial T }_{i} + \nu_{k,i} \frac{\partial R }{\partial T }_{i} c_{i}\right) \;,
\end{equation}
other species:
\begin{equation}
 \label{e:dwdot_dnj}
 \frac{\partial \dot{\omega} }{\partial n[j] }_{k} = \sum_{i=1}^{\nr} \left(\nu_{k,i} R_{i} \frac{\partial c }{\partial n[j] }_{i} + \nu_{k,i} \frac{\partial R }{\partial n[j] }_{i} c_{i}\right)
\end{equation}
and the state parameter:
\begin{subequations}
 \label{e:dwdot_de}
 \begin{align}
  \frac{\partial \dot{\omega} }{\partial V }_{k} &= \sum_{i=1}^{\nr} \left(\nu_{k,i} R_{i} \frac{\partial c }{\partial V }_{i} + \nu_{k,i} \frac{\partial R }{\partial V }_{i} c_{i}\right)\dconp,\\
  \frac{\partial \dot{\omega} }{\partial P }_{k} &= \sum_{i=1}^{\nr} \left(\nu_{k,i} R_{i} \frac{\partial c }{\partial P }_{i} + \nu_{k,i} \frac{\partial R }{\partial P }_{i} c_{i}\right)\dconv
 \end{align}
\end{subequations}

\subsection{Rate of progress derivatives}
\subsubsection{Temperature derivative}
Next, the rate of progress derivatives are evaluated, first with respect to temperature:
\begin{equation}
 \frac{\partial R }{\partial T }_{i} = \frac{\partial}{\partial T}\left({k_f}_{i} \prod_{k=1}^{\ns} [C]_{k}^{\nu^{\prime}_{k,i}}\right)
\end{equation}
the derivative of the forward reaction rate coefficient is:
\begin{equation}
 \label{e:dkf_dt}
 \frac{\text{d} {k_f} }{\text{d} T }_{i} = \frac{{k_f}_{i}}{T} \left(\beta_{i} + \frac{{E_{a}}_{i}}{T \Ru}\right)
\end{equation}
While evaluating derivatives for the Jacobian, it is important to expand terms---e.g., concentration, rate of production, etc.---related to the last species to obtain the correct derivative, in this case using Eq.~\cref{e:last_spec_conc}:
\begin{equation}
 {R_f} = \left(\left(- \sum_{k=1}^{-1 + \ns} [C]_{k} + \frac{P}{T \Ru}\right)^{\nu^{\prime}_{\ns,i}}\right) {k_f}_{i} \prod_{k=1}^{-1 + \ns} [C]_{k}^{\nu^{\prime}_{k,i}}
\end{equation}
The derivative of a species concentration with respect to temperature is zero for both \conp~and \conv:
\begin{align}
 \frac{\partial [C]_{k}}{\partial T} &= \frac{\partial}{\partial T} \left(\frac{n_k}{V}\right), \quad \left\{k \ne \ns\right\} \nonumber \\
				     &= 0
\end{align}
as the volume is either constant (\conp) or a state variable and a function of time only (\conv).
Hence, the full derivative of the forward rate of progress is:
\begin{equation}
 \label{e:dropfdt_1}
 \frac{\partial {R_f} }{\partial T }_{i} = \frac{\text{d} {k_f} }{\text{d} T }_{i} \prod_{k=1}^{\ns} [C]_{k}^{\nu^{\prime}_{k,i}} - \frac{[C] \nu^{\prime}_{\ns,i}}{T} [C]_{\ns}^{\nu^{\prime}_{\ns,i} - 1} {k_f}_{i} \prod_{k=1}^{-1 + \ns} [C]_{k}^{\nu^{\prime}_{k,i}}
\end{equation}
By defining a temporary variable (for species $l = \left\{j, \ns\right\}$) $S^{\prime}_{l}$:
\begin{equation}
 \label{e:s_temp}
 S^{\prime}_{l} = \nu^{\prime}_{l,i} [C]_{l}^{\nu^{\prime}_{l,i} - 1} \prod_{\substack{1 \leq l \leq l - 1\\l + 1 \leq l \leq \ns}} [C]_{l}^{\nu^{\prime}_{l,i}}
\end{equation}
and using~\cref{e:dkf_dt}, Eq.~\cref{e:dropfdt_1} can be simplified to:
\begin{equation}
 \label{e:dropf_dt}
 \frac{\partial {R_f} }{\partial T }_{i} = - \frac{[C] S^{\prime}_{\ns}}{T} {k_f}_{i} + \frac{{R_f}_{i}}{T} \left(\beta_{i} + \frac{{E_{a}}_{i}}{T \Ru}\right)
\end{equation}

Starting from~\cref{e:ropr} and applying the same expansion of the last species' concentration as previously, the temperature derivative of the reverse rate of progress is:
\begin{equation}
 \frac{\partial {R_r} }{\partial T }_{i} = \frac{\text{d} {k_r} }{\text{d} T }_{i} \prod_{k=1}^{\ns} [C]_{k}^{\nu^{\prime\prime}_{k,i}} - \frac{[C] \nu^{\prime\prime}_{\ns,i}}{T} [C]_{\ns}^{\nu^{\prime\prime}_{\ns,i} - 1} {k_f}_{i} \prod_{k=1}^{-1 + \ns} [C]_{k}^{\nu^{\prime\prime}_{k,i}}
\end{equation}
Next, Eq.~\cref{e:kr} is considered to obtain the temperature derivative of the non-explicit reversible reaction rate coefficient:
\begin{equation}
 \frac{\text{d} {k_r} }{\text{d} T }_{i} = \left(- \frac{1}{{K_c}_{i}} \frac{\text{d} {K_c} }{\text{d} T }_{i} + \frac{1}{T} \left(\beta_{i} + \frac{{E_{a}}_{i}}{T \Ru}\right)\right) {k_r}_{i}
\end{equation}
The temperature derivative of the equilibrium constant is:
\begin{equation}
 \frac{\text{d} {K_c} }{\text{d} T }_{i} = {K_c}_{i} \sum_{k=1}^{\ns} \nu_{k,i} \frac{\text{d} B }{\text{d} T }_{k} \;,
\end{equation}
with:
\begin{equation}
 \frac{\text{d} B }{\text{d} T }_{k} = T \left(T \left(\frac{T a_{k,4}}{5} + \frac{a_{k,3}}{4}\right) + \frac{a_{k,2}}{3}\right) + \frac{a_{k,1}}{2} + \frac{1}{T} \left(a_{k,0} - 1 + \frac{a_{k,5}}{T}\right)
\end{equation}
giving:
\begin{equation}
\label{e:dkr_dt}
\frac{\text{d} {k_r} }{\text{d} T }_{i} = \left(- \sum_{k=1}^{\ns} \nu_{k,i} \frac{\text{d} B }{\text{d} T }_{k} + \frac{1}{T} \left(\beta_{i} + \frac{{E_{a}}_{i}}{T \Ru}\right)\right) {k_r}_{i} \;,
\end{equation}
and using an temporary value $S^{\prime\prime}_{\ns}$---defined analagously to Eq.~\cref{e:s_temp} for the reverse direction---we obtain:
\begin{equation}
\label{e:dropr_dt}
\frac{\partial {R_r} }{\partial T }_{i} = \left(- \sum_{k=1}^{\ns} \nu_{k,i} \frac{\text{d} B }{\text{d} T }_{k} + \frac{1}{T} \left(\beta_{i} + \frac{{E_{a}}_{i}}{T \Ru}\right)\right) {R_r}_{i} - \frac{[C] S^{\prime\prime}_{\ns}}{T} {k_r}_{i}
\end{equation}

Finally, combining~\cref{e:dropf_dt,e:dropr_dt} gives the total temperature derivative of the net rate of progress:
\begin{align}
 \label{e:drop_dt}
 \frac{\partial R }{\partial T }_{i} =& \frac{{R_f}_{i}}{T} \left(\beta_{i} + \frac{{E_{a}}_{i}}{T \Ru}\right) - \left(- \sum_{k=1}^{\ns} \nu_{k,i} \frac{\text{d} B }{\text{d} T }_{k} + \frac{1}{T} \left(\beta_{i} + \frac{{E_{a}}_{i}}{T \Ru}\right)\right) {R_r}_{i} \nonumber \\
				      &\qquad + \frac{[C] S^{\prime\prime}_{\ns}}{T} {k_r}_{i} - \frac{[C] S^{\prime}_{\ns}}{T} {k_f}_{i}
\end{align}
\subsubsection{Molar derivative}
\label{s:rop_molar}
The derivative of the forward rate of progress with respect to the amount of moles of a species $j$ is:
\begin{equation}
 \frac{d}{d n_{k}} {R_f} = \left(\frac{\partial}{\partial n_{j}} \prod_{k=1}^{\ns} [C]_{k}^{\nu^{\prime}_{k,i}}\right) {k_f}_{i} \;.
\end{equation}
The derivative of a species concentration with respect to moles of a species $j$ is given as:
\begin{equation}
 \frac{\partial [C_k]}{\partial n_j} =
 \begin{dcases}
 \frac{\delta_{j k}}{V} & k \ne \ns \\
 -\frac{1}{V} & k = \ns
 \end{dcases}
\end{equation}
Or put succinctly:
\begin{equation}
\label{e:dck_dnj}
\frac{\partial [C_k]}{\partial n_j} = \frac{\delta_{j k} - \delta_{\ns k}}{V} \;.
\end{equation}
Using~\cref{e:dck_dnj} the molar derivative of the forward rate of progress is found as:
\begin{equation}
 \frac{\partial {R_f} }{\partial n[j] }_{i} = {k_f}_{i} \sum_{k=1}^{\ns} \left(- \frac{\delta_{\ns k}}{V} + \frac{\delta_{j k}}{V}\right) \nu^{\prime}_{k,i} [C]_{k}^{\nu^{\prime}_{k,i} - 1} \prod_{\substack{1 \leq l \leq k - 1\\k + 1 \leq l \leq \ns}} [C]_{l}^{\nu^{\prime}_{l,i}}
\end{equation}
with the same temporary variables $S^{\prime}_{j}$ and $S^{\prime}_{\ns}$ as defined previously, this can be simplified to:
\begin{equation}
 \label{e:dropf_dnj}
 \frac{\partial {R_f} }{\partial n[j] }_{i} = \frac{{k_f}_{i}}{V} \left(- S^{\prime}_{\ns} + S^{\prime}_{j}\right)
\end{equation}
and by the same process, the reverse rate of progress derivative is found to be:
\begin{equation}
 \label{e:dropr_dnj}
 \frac{\partial {R_r} }{\partial n[j] }_{i} = \frac{{k_r}_{i}}{V} \left(- S^{\prime\prime}_{\ns} + S^{\prime\prime}_{j}\right)
\end{equation}
and the net rate of progress molar derivative:
\begin{equation}
 \frac{\partial R }{\partial n[j] }_{i} = - \frac{{k_r}_{i}}{V} \left(- S^{\prime\prime}_{\ns} + S^{\prime\prime}_{j}\right) + \frac{{k_f}_{i}}{V} \left(- S^{\prime}_{\ns} + S^{\prime}_{j}\right)
\end{equation}

\subsubsection{Parameter derivatives}
The derivatives of the forward and reverse rates of progress with respect to the volume (\conp) or pressure (\conv) largely follow the same format of the previous sections.
First, the derivative of the concentration of a species $k$ with respect to the parameter is:

\begin{subequations}
 % based on https://tex.stackexchange.com/a/239764/56227
 % the idea is to take the longest case (1.b -- i.e., no hphantom) and use this
 % to make a box around all cases such that the alignment works
 \begin{align}
 \frac{\partial [C] }{\partial V }_{k} &=
 \begin{dcases}
  \mathrlap{-\frac{[C]_{k}}{V}}\hphantom{\frac{1}{V} \sum_{k=1}^{-1 + \ns} [C]_{k}} & k \ne \ns \\
  \frac{1}{V} \sum_{k=1}^{-1 + \ns} [C]_{k} & k = \ns
 \end{dcases}&\dconp \\
 \frac{\partial [C]_{k} }{\partial P } &=
 \begin{dcases}
  \mathrlap{\hphantom{0}0}\hphantom{\frac{1}{V} \sum_{k=1}^{-1 + \ns} [C]_{k}} & k \ne \ns \\
  \mathrlap{\frac{1}{T \Ru}}\hphantom{\frac{1}{V} \sum_{k=1}^{-1 + \ns} [C]_{k}} & k = \ns
 \end{dcases}&\dconv
 \end{align}
\end{subequations}

Following the outline of~\cref{s:rop_molar}, the forward rate of progress derivatives are:
\begin{subequations}
 \begin{align}
  \frac{\partial {R_f} }{\partial V }_{i} &= \frac{[C] S^{\prime}_{\ns}}{V} {k_f}_{i} - \frac{{R_f}_{i}}{V} \sum_{k=1}^{\ns} \nu^{\prime}_{k,i}&\dconp\\
  \frac{\partial {R_f} }{\partial P }_{i} &= \frac{S^{\prime}_{\ns} {k_f}_{i}}{T \Ru}&\dconv
 \end{align}
\end{subequations}
and similarly, the reverse rate of progress derivatives:
\begin{subequations}
 \begin{align}
  \frac{\partial {R_r} }{\partial V }_{i} &= \frac{[C] S^{\prime\prime}_{\ns}}{V} {k_r}_{i} - \frac{{R_r}_{i}}{V} \sum_{k=1}^{\ns} \nu^{\prime\prime}_{k,i}&\dconp\\
  \frac{\partial {R_r} }{\partial P }_{i} &= \frac{S^{\prime\prime}_{\ns} {k_r}_{i}}{T \Ru}&\dconv
 \end{align}
\end{subequations}
giving finally:
\begin{subequations}
 \label{e:dropi_de}
 % linebreak and parenthesis sizing:
 % https://tex.stackexchange.com/questions/49890/linebreak-between-left-and-right/49895
 \begin{align}
  \frac{\partial {R} }{\partial V }_{i} &= \frac{1}{V}\left(\vphantom{{R_r}_{i} \sum_{k=1}^{\ns} \nu^{\prime\prime}_{k,i}} [C]\left(S^{\prime}_{\ns} {k_f}_{i} - S^{\prime\prime}_{\ns} {k_r}_{i}\right) + \right.& \nonumber\\
  &\left. \quad\left({R_r}_{i} \sum_{k=1}^{\ns} \nu^{\prime\prime}_{k,i} - {R_f}_{i} \sum_{k=1}^{\ns} \nu^{\prime}_{k,i}\right) \right)&\dconp\\
  \frac{\partial {R} }{\partial P }_{i} &= \frac{1}{T \Ru} \left(S^{\prime}_{\ns} {k_f}_{i} - S^{\prime\prime}_{\ns} {k_r}_{i}\right)&\dconv
 \end{align}
\end{subequations}

\subsection{Pressure modification\slash Falloff function derivatives}




\subsection{Final Jacobian entries}
\label{s:jac_final}

\printbibliography 

\end{document}
